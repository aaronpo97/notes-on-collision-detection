\documentclass{report}

% Geometry and layout
\usepackage[margin=3.3cm]{geometry}
\usepackage[skip=10pt plus1pt, indent=0em]{parskip}
\usepackage{multicol}
\usepackage{titlesec}
\usepackage{csquotes}
\usepackage{float}
\usepackage{palatino}

\usepackage{fancyhdr}

% Links and references
\usepackage{hyperref}

% Math and symbols
\usepackage{amsmath}
\usepackage{amssymb}
\usepackage{amsthm}

% Graphics
\usepackage{graphicx}
\usepackage{svg}
\usepackage{tikz}

% Code listings
\usepackage{listings}
\usepackage{color}
\definecolor{dkgreen}{rgb}{0,0.6,0}
\definecolor{gray}{rgb}{0.5,0.5,0.5}
\definecolor{mauve}{rgb}{0.58,0,0.82}

% Frames and boxes
\usepackage{mdframed}

% Theorems and remarks
\newtheorem*{remark}{Remark}

% Text
\makeatletter
\renewcommand\maketitle{
    \begin{center}
        \vspace*{12em}
        {\textbf{\LARGE{\@title}}} \\
        \vspace{1em}
        {\textbf{\Large{\@author}}} \\
        \vspace{1em}
        {\textbf{\large{\@date}}} \\
        \vspace{2em}
    \end{center}
}

% Custom commands
% \renewcommand{\land}{\,\,\textrm{\textbf{AND}}\,\,}
\newcommand{\subsubsubsection}[1]{\paragraph{#1}\mbox{}\\}
\newcommand{\drawaxesgrid}[2]{
    % Draw x and y axes with arrows
    \draw[thick,->] (0,0) -- (#1,0) node[anchor=west] {\scriptsize x};
    \draw[thick,->] (0,0) -- (0,#2) node[anchor=north] {\scriptsize y};

    % Calculate the number of ticks based on sizes
    \pgfmathtruncatemacro{\xtickstop}{#1-1}
    \pgfmathtruncatemacro{\ytickstop}{#2-1}

    % a point at the origin (0, 0) 
    \filldraw [black] (0, 0) circle (5pt) {};

    % Draw x-axis tick marks and labels
    \foreach \x in {1,...,\xtickstop}
    \draw (\x,0.1) -- (\x,-0.1) node[anchor=south] {\scriptsize \x};

    % Draw y-axis tick marks and labels
    \foreach \y in {1,...,\ytickstop}
    \draw (0.1,\y) -- (-0.1,\y) node[anchor=east] {\scriptsize \y};

    % Draw light gray grid lines
    \foreach \x in {1,...,\xtickstop}
    \draw[lightgray,dashed] (\x,0) -- (\x,#2);
    \foreach \y in {1,...,\ytickstop}
    \draw[lightgray,dashed] (0,\y) -- (#1,\y);
}

\author{Aaron Po}
\title{Collision Detection and Collision Resolution}
\date{\today}
\begin{document}
\maketitle

\tableofcontents
\section{Introduction}
Collision detection is a fundamental concept in game development, determining
when two or more objects intersect within a game world. It is crucial for many
game mechanics, including physics, AI behavior, and rendering effects.
Essentially, collision detection answers the question:

\begin{displayquote}
    \textit{Given two entities, each with a position and shape, do they intersect? If so,
        how can we resolve the collision?}
\end{displayquote}

In this document, I will discuss various methods for collision detection and
collision resolution using bounding shapes. These methods and techniques are
essential for the creation of efficient and engaging games. This document is
based on course material from COMP 4300 at the Memorial University of
Newfoundland as well as my own personal study.


\section{Introduction}
Collision detection is a fundamental concept in game development, determining
when two or more objects intersect within a game world. It is crucial for many
game mechanics, including physics, AI behavior, and rendering effects.
Essentially, collision detection answers the question:

\begin{displayquote}
    \textit{Given two entities, each with a position and shape, do they intersect? If so,
        how can we resolve the collision?}
\end{displayquote}

In this document, I will discuss various methods for collision detection and
collision resolution using bounding shapes. These methods and techniques are
essential for the creation of efficient and engaging games. This document is
based on course material from COMP 4300 at the Memorial University of
Newfoundland as well as my own personal study.



\section*{Appendix}

The following code snippets are implementations of the \texttt{Vec2} class in
C++. The \texttt{Vec2} class is a simple two-dimensional vector class that
provides basic vector operations such as addition, subtraction, multiplication,
and division. The class is used in this document for various collision
detection and resolution algorithms.

\subsection*{Vec2.h}
\vspace{1em}
\begin{mdframed}[linecolor=black!30!white,linewidth=.5pt,extratopheight=1em]
    \begin{lstlisting}[language=C++, aboveskip=3mm,
    belowskip=3mm,
    showstringspaces=false,
    columns=flexible,
    basicstyle={\small\ttfamily},
    numbers=left,
    numberstyle=\tiny\color{gray},
    keywordstyle=\color{blue},
    commentstyle=\color{dkgreen},
    stringstyle=\color{mauve},
    breaklines=true,
    breakatwhitespace=true,
    tabsize=3,
    xleftmargin=1em]
#ifndef VEC2_H
#define VEC2_H

class Vec2 {
public:
};
    double x, y;

    Vec2(float x = 0, float y = 0);

    bool operator==(const Vec2 &rhs) const;
    bool operator!=(const Vec2 &rhs) const;

    Vec2 operator-(const Vec2 &rhs) const;
    Vec2 operator+(const Vec2 &rhs) const;
    Vec2 operator*(const float val) const;
    Vec2 operator/(const float val) const;

    void operator+=(const Vec2 &rhs);
    void operator-=(const Vec2 &rhs);
    void operator*=(const float val);
    void operator/=(const float val);

    Vec2  normalize();
    float length() const;
};

#endif // VEC2_H
\end{lstlisting}
\end{mdframed}

\subsection*{Vec2.cpp}
\vspace{1em}
\begin{mdframed}[linecolor=black!30!white,linewidth=.5pt,extratopheight=1em]
    \begin{lstlisting}[language=C++, aboveskip=3mm,
    belowskip=3mm,
    showstringspaces=false,
    columns=flexible,
    basicstyle={\small\ttfamily},
    numbers=left,
    numberstyle=\tiny\color{gray},
    keywordstyle=\color{blue},
    commentstyle=\color{dkgreen},
    stringstyle=\color{mauve},
    breaklines=true,
    breakatwhitespace=true,
    tabsize=3,
    xleftmargin=1em]

#include "Vec2.h"
#include <cmath>

Vec2::Vec2(float x, float y) : x(x), y(y) {}

bool Vec2::operator==(const Vec2 &rhs) const { return rhs.x == x && rhs.y == y;
    }

bool Vec2::operator!=(const Vec2 &rhs) const { return !(*this == rhs); }

Vec2 Vec2::operator-(const Vec2 &rhs) const { return Vec2(x - rhs.x, y -
        rhs.y); }

Vec2 Vec2::operator+(const Vec2 &rhs) const { return Vec2(x + rhs.x, y +
        rhs.y); }

Vec2 Vec2::operator*(const float val) const { return Vec2(x * val, y * val); }

Vec2 Vec2::operator/(const float val) const { return Vec2(x / val, y / val); }

void Vec2::operator+=(const Vec2 &rhs) { x += rhs.x; y += rhs.y; }

void Vec2::operator-=(const Vec2 &rhs) { x -= rhs.x; y -= rhs.y; }

void Vec2::operator*=(const float val) { x *= val; y *= val; }

void Vec2::operator/=(const float val) { x /= val; y /= val; }

float Vec2::length() const { return std::sqrt(x * x + y * y); }

Vec2 Vec2::normalize() {
        float len = length();
        if (len > 0) { // Avoid division by zero
                x /= len;
                y /= len;
            }

        return *this; }

\end{lstlisting}
\end{mdframed}
\end{document}
