\chapter{Introduction}
Collision detection is a fundamental concept in game development, determining
when two or more objects intersect within a game world. It is crucial for many
game mechanics, including physics, AI behavior, and rendering effects.
Essentially, collision detection answers the question:

\begin{displayquote}
    \textit{Given two entities, each with a position and shape, do they intersect? If so,
        how can we resolve the collision?}
\end{displayquote}

In this document, I will discuss various methods for collision detection and
collision resolution using bounding shapes such as circles and rectangles.
These methods and techniques are essential for the creation of efficient and
engaging games. This document is based on course material from COMP 4300 at the
Memorial University of Newfoundland as well as my own personal study. I hope
you find this document helpful and informative.

\section{Objectives}
\begin{itemize}
    \item Understand the importance of collision detection in game development.
    \item Learn about different types of bounding shapes used in collision detection.
    \item Explore methods for detecting collisions between entities.
    \item Investigate techniques for resolving collisions between entities.
    \item Implement collision detection and resolution algorithms in a game engine.
\end{itemize}

\section{Prerequisites}
To fully understand the content in this document, you should have a basic
understanding of game development concepts, including vectors, matrices, and
physics. Familiarity with the C++ programming language is also recommended.

