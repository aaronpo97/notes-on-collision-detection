\chapter{Collision Resolution}
In the next chapter, we will discuss collision resolution. Collision resolution
involves determining the outcome when two entities collide. This can include
adjusting their velocities, positions, or even removing them from the
simulation. Throughout this chapter, we will explore different methods of
collision resolution and how they can be implemented in a game engine.

Before diving into collision resolution, it's important to understand the
concept of an Entity. An Entity is an object within the game world that
consists of various components, such as its transform (position, rotation,
scale, velocity), bounding shape (circle, rectangle), and other properties like
its name. In this chapter, our focus will be on resolving collisions between
entities based on their bounding shapes.

\section{Circle-Circle Collision Resolution}
In the previous chapter, we discussed how to detect collisions between two
circles. In this section, we will explore how to resolve these collisions.

\subsection{Adjusting Velocities}
When two entity circles collide in a game, we can adjust their velocities to
simulate the effect of the collision. A simple way to do this is to simply
reverse the direction of the velocity vector for each circle.

Assume the two entities have a velocity vector $v_1$ and $v_2$, respectively.
After the collision, we can update their velocities as follows:
\begin{equation}
    \begin{aligned}
        v_1 & = v_1 * -1  \\
        v_2 & = -v_2 * -1
    \end{aligned}
\end{equation}

This will cause the two circles to bounce off each other in the opposite
direction.

